\section{Skillchecks}
To determine wether someone succeeds or fails at a specific task, this ruleset uses a d100 or d\% system. That means you throw a 100 sided die (a so called ''d100'') and try to role a lower or equal number to your skill level to succeed, if you roll a higher number you fail.

Since there aren't many d100 around, it is recommended that you roll two 10 sided dice (d10) and declare one as the tens and the other as the ones (before you throw them, of course).
\subsection{Attribute check}
Sometimes you have to check for an attribute directly. In this case you multiply the corresponding attribute by ten and role a d100\footnote{You could just role a d10 against the attribute directly but there are some events and perks that will modify your roll by numbers not multiple to 10}. Again, the check succeeds if you roll lower or equal to the value and fails if you roll higher.
\subsection{Critical successes and \mbox{failures}}
Whenever you roll under or equal to a $\frac{1}{10}$ of your original target number, you score a critical success. Critical successes give your weapons more damage, your containers more and more valuable loot and gives you extra options during dialogues, simply put, it makes your success even more better.

In contrary, whenever you roll higher than 89 + the LUCK of the character you are rolling for, it is a critical failure. A critical failure will not only have grave consequences but also prevent any further attempt to succeed in the same matter, e.g. a lock or your gun will jam, a terminal locks down and alarms are sounded.
\subsection{Modifiers}
Sometimes, you find yourself at a task that is either easier or more difficult than what has been encountered on average, for example picking a state-of-the-art lock or persuading a very naive trader to give you a discount or tell you a secret. In these cases the GM may modify a roll to better fit the situation, he will tell you to either add or subtract a certain number to your roll. If, with your added modifiers, you still roll under, equal or above your target number, you succeed or fail at your task as usual.

To ease a task you subtract and to complicate it you add appropriate modifiers.
\begin{Example}
Little Pip wants to pick a lock of a safe of medium difficulty. Her lockpicking skill is 100 and her luck is 5, so with a roll of more than 94, the bobbypin snaps and gets stuck in the lock, never giving its secrets to the Stable Dweller. With a roll of 94 or less, she opens the lock like any other she had encountered until yet, with a roll of 10 or less, she scores a critical success, opening the lock with ease and revealing even more and more valuable loot than with a normal success. She rolls a 19, adding 50 to the roll because of its complexity she still pries open the lock and stuffs her saddlebags with meager, normal contents of a safe (usually some degraded paper, pre-war money and 10mm bullets).
\end{Example}