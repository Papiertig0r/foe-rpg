\section{Skillchecks}
To determine wether someone succeeds or fails at a specific task, this ruleset uses a d100 or d\% system. That means you throw a 100 sided die (a so called ''d100'') and try to role a lower or equal number to your skill level to succeed, if you roll a higher number you fail.

Since there aren't many d100 around, it is recommended that you roll two 10 sided dice (d10) and declare one as the tens and the other as the ones (before you throw them, of course).
\subsection{Attribute check}
Sometimes you have to check for an attribute directly. In this case you multiply the corresponding attribute by ten and role a d100\footnote{You could just role a d10 against the attribute directly but there are some events and perks that will modify your roll by numbers not multiple to 10}. Again, the check succeeds if you roll lower or equal to the value and fails if you roll higher.
\subsection{Critical successes and \mbox{failures}}
Whenever you roll under or equal to a $\frac{1}{10}$ of your original target number, you score a critical success. Critical successes give your weapons more damage, your containers more and more valuable loot and gives you extra options during dialogues, simply put, it makes your success even more better.

In contrary, whenever you roll higher than 89 + the LUCK of the character you are rolling for, it is a critical failure. A critical failure will not only have grave consequences but also prevent any further attempt to succeed in the same matter, e.g. a lock or your gun will jam, a terminal locks down and alarms are sounded.
\subsection{Modifiers}
You sometimes may find yourself at a task that is either easier or more difficult than what has been encountered on average, for example picking a state-of-the-art lock or persuading a very naive trader to give you a discount or tell you a secret. In these cases the GM may modify a roll to better fit the situation, he will tell you to either add or subtract a certain number to your roll. If, with your added modifiers, you still roll under, equal or above your target number, you succeed or fail at your task as usual.

To ease a task you subtract and to complicate it you add appropriate modifiers.
\begin{Example}
Little Pip wants to pick a lock of a safe of medium difficulty. Her lockpicking skill is 100 and her luck is 5, so with a roll of more than 94, the bobbypin snaps and gets stuck in the lock, never giving its secrets to the Stable Dweller. With a roll of 94 or less, she opens the lock like any other she had encountered until yet, with a roll of 10 or less, she scores a critical success, opening the lock with ease and revealing even more and more valuable loot than with a normal success. She rolls a 19, adding 50 to the roll because of its complexity she still pries open the lock and stuffs her saddlebags with meager, normal contents of a safe (usually some degraded paper, pre-war money and 10mm bullets).
\end{Example}

\section{Combat}
Fighting is an important part of a life in the wastelands - you seldom find somepony or someone that hadn't engaged in combat at least once in his life.
\subsection{Initiative}
Most rulesets use a round-based combat system where every player gets a turn until every player had one - the first player beeing determined by agility or an initiative roll of some sort. While both these concepts are coming to use in this system, the order of the players varies from round to round basing on the actions the players take. For tracking these orders, you need the round counter from the Appendix. You place a marker for every character partaking in the battle on its corresponding AGI score and place the active player marker on the field with the character with the highest AGI. After a turn, the active character will be set back by a number of fields equal to the cost of the action they performed and the active player marker goes to the next character in line. Are two characters on the same field, the one with the higher AGI acts first. If they have the same AGI, they simply act simultaneously. 
\subsection{Actions}
During his or her turn, a player can perform one of the following actions:
\begin{description}
\item[Move.] Move up to your moving speed or fly, if you have wings. Moving will put you one field further on the round counter. Base attribute for moving and flying is END.
\item[Hold your action.] You simply do nothing and advance one field on the round counter.
\item[Attack.] Use the weapon you are wielding to hurt somepony else.
\item[Reload.] Get some more ammunition into your weapon. The base attribute for reloading is AGI.
\item[Get an item.] Bring an item from your saddlebag into your hooves or vice versa or draw or holster a weapon. The base attribute for getting an item is AGI.
\end{description}
\subsubsection{Multiple Actions}
It is possible to perform more than one action per round but only as many actions as you have AGI score. Each action will put you forward on the round counter as usual, but for every action after your first, you recieve an additional, accumulating penalty of 20! You can also only perform actions until you fail one, beginning with the first one. Since most actions don't have to be rolled for when using them as your free (first) action, they have a base attribute attached on which you roll when using them as a subsequent action.